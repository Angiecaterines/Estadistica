\documentclass[11pt,letterpaper]{article}\usepackage[]{graphicx}\usepackage[]{color}
% maxwidth is the original width if it is less than linewidth
% otherwise use linewidth (to make sure the graphics do not exceed the margin)
\makeatletter
\def\maxwidth{ %
  \ifdim\Gin@nat@width>\linewidth
    \linewidth
  \else
    \Gin@nat@width
  \fi
}
\makeatother

\definecolor{fgcolor}{rgb}{0.345, 0.345, 0.345}
\newcommand{\hlnum}[1]{\textcolor[rgb]{0.686,0.059,0.569}{#1}}%
\newcommand{\hlstr}[1]{\textcolor[rgb]{0.192,0.494,0.8}{#1}}%
\newcommand{\hlcom}[1]{\textcolor[rgb]{0.678,0.584,0.686}{\textit{#1}}}%
\newcommand{\hlopt}[1]{\textcolor[rgb]{0,0,0}{#1}}%
\newcommand{\hlstd}[1]{\textcolor[rgb]{0.345,0.345,0.345}{#1}}%
\newcommand{\hlkwa}[1]{\textcolor[rgb]{0.161,0.373,0.58}{\textbf{#1}}}%
\newcommand{\hlkwb}[1]{\textcolor[rgb]{0.69,0.353,0.396}{#1}}%
\newcommand{\hlkwc}[1]{\textcolor[rgb]{0.333,0.667,0.333}{#1}}%
\newcommand{\hlkwd}[1]{\textcolor[rgb]{0.737,0.353,0.396}{\textbf{#1}}}%
\let\hlipl\hlkwb

\usepackage{framed}
\makeatletter
\newenvironment{kframe}{%
 \def\at@end@of@kframe{}%
 \ifinner\ifhmode%
  \def\at@end@of@kframe{\end{minipage}}%
  \begin{minipage}{\columnwidth}%
 \fi\fi%
 \def\FrameCommand##1{\hskip\@totalleftmargin \hskip-\fboxsep
 \colorbox{shadecolor}{##1}\hskip-\fboxsep
     % There is no \\@totalrightmargin, so:
     \hskip-\linewidth \hskip-\@totalleftmargin \hskip\columnwidth}%
 \MakeFramed {\advance\hsize-\width
   \@totalleftmargin\z@ \linewidth\hsize
   \@setminipage}}%
 {\par\unskip\endMakeFramed%
 \at@end@of@kframe}
\makeatother

\definecolor{shadecolor}{rgb}{.97, .97, .97}
\definecolor{messagecolor}{rgb}{0, 0, 0}
\definecolor{warningcolor}{rgb}{1, 0, 1}
\definecolor{errorcolor}{rgb}{1, 0, 0}
\newenvironment{knitrout}{}{} % an empty environment to be redefined in TeX

\usepackage{alltt} 
\usepackage[utf8]{inputenc} 
\usepackage[spanish, es-tabla, es-nodecimaldot]{babel} % Formato a español
\usepackage[T1]{fontenc}    % Permite utilizar otras tipografías
\usepackage[vmargin=2.5cm,hmargin=2cm]{geometry}
\usepackage{multicol}   % Unir columnas en tablas y formato a dos columnas
\usepackage{multirow}   % Unir filas en tablas
\usepackage{graphicx}   % Necesario para insertar gráficas
\usepackage{float}      % Corregir ubicación de imágenes y tablas
%\usepackage{subfigure} % Insertar subfiguras
%\usepackage{url}       % Hiervínculo a direcciones URL
\usepackage{listings}
\usepackage{flushend}
\usepackage{minted}
\usepackage[none]{hyphenat} % Permite utilizar el comando \sloppy
\usepackage[small]{caption}	% Reduce el tamaño de letra utilizado en los pies de figura.
\usepackage{hyperref}   % Agrega enlaces internos de las secciones, figuras y tablas.
\usepackage{color}      % Definición de colores
    \hypersetup{colorlinks=true, linkcolor=[rgb]{0,0,1}, citecolor=[rgb]{0,0,1}}
\usepackage{xcolor}		% Permite definir un color para utilizarlo dentro del documento.
    \definecolor{gris}{RGB}{70,70,70}	% Definiendo el color gris
    \definecolor{negro}{RGB}{40,40,40}		% Definiendo el color negro

%%%%%%%% Modificación de los espacios de los títulos de secciones %%%%%%%%%%
\usepackage{titlesec}		% Permite reconfigurar  los títulos de las secciones y subsecciones
\renewcommand\thesection{\Roman{section}}	% Numeración romana en las secciones
\renewcommand\thesubsection{\Roman{subsection}}		% Numeración romana en las subsecciones
\titlespacing*{\section}{0pt}{2.5mm}{0mm}	% Espaciado del título {espacio izquierdo}{arriba del título}{abajo del título}
\titleformat{\section}[block]{\large\scshape\centering}{\thesection.}{1em}{}	% Espaciado del título de las secciones
\titleformat{\subsection}[block]{\large}{\thesubsection.}{1em}{}				% Espaciado del título de las subsecciones
%%%%%%%%%%%%%%%%%%%%%%%%%%%%%%%%%%%%%%%%%%%%%%%%%%%%%%%%%%%%%%%%%%%%%%%%%%%%%%

%Se define un comando \colorhrule para hacer líneas horizontales de color con 3 argumentos: color, largo, ancho.
\newcommand{\colorhrule}[3]{\begingroup\color{#1}\rule{#2}{#3}\endgroup}

\setlength{\intextsep}{1mm} % Distancia superior e inferior en objetos flotantes
\setlength{\columnsep}{5mm} % Separación entre columnas del documento
\IfFileExists{upquote.sty}{\usepackage{upquote}}{}
\begin{document}
\sloppy     % Evita que las palabras se corten al saltar de línea.
\begin{center}
\begin{tabular}{cc}
\multirow{2}{3.5cm}{\includegraphics[width=4cm]{UEBlogo.png}}	& \huge{\textsc{\textbf{Universidad El Bosque}}}\\ %\vspace{5mm}
 & \scriptsize{\textsc{FACULTAD DE CIENCIAS}}\\[5mm]
 & \Large{\textsf{\textbf{Modelos ML para regresión y clasificación}}}\\
 & \small{\textsf{Angie Caterine Sarmiento Gonzalez [1233507154]}}\\ \vspace{5mm}
 %& \small{\textsf{Nombre completo de alumno 2 [Código]}}\\
 & \small{\textsc{Estadística $|$ Minería de Datos}}\\
 & \today\\
\end{tabular}
\end{center}
	

\begin{center}
\colorhrule{black}{16.5cm}{1.2pt}
\end{center}

\section*{\textbf{Actividad}}

Construir y validar el modelo de regresión más potente para predecir el precio de venta Price de un automóvil nuevo con base en las variables predictoras $X1$=KM (kilometraje), $X2$=Age (años de uso) y $X3$=Weight (peso)

\section*{\textbf{Pasos a seguir}}
\begin{enumerate}

    \item Seleccione en un mismo data frame las variables de interés.

    
\begin{knitrout}
\definecolor{shadecolor}{rgb}{0.969, 0.969, 0.969}\color{fgcolor}\begin{kframe}
\begin{verbatim}
## # A tibble: 6 x 4
##      KM   Age Weight Price
##   <dbl> <dbl>  <dbl> <dbl>
## 1 46986    23   1165 13500
## 2 72937    23   1165 13750
## 3 41711    24   1165 13950
## 4 48000    26   1165 14950
## 5 38500    30   1170 13750
## 6 61000    32   1170 12950
\end{verbatim}
\end{kframe}
\end{knitrout}


\item Construya una conjunto de entrenamiento (75\%) y otro de prueba (25\%). Tome la semilla 12345
    
\begin{knitrout}
\definecolor{shadecolor}{rgb}{0.969, 0.969, 0.969}\color{fgcolor}\begin{kframe}
\begin{alltt}
\hlcom{# Seleccionar siempre la misma partición}
\hlkwd{set.seed}\hlstd{(}\hlnum{12345}\hlstd{)}

\hlcom{# Muestra aleatoria del 75% de las filas del conjunto "datos" para el conjunto de entrenamiento}
\hlstd{train.filas}\hlkwb{<-}\hlkwd{sample}\hlstd{(}\hlkwc{x}\hlstd{=}\hlkwd{row.names}\hlstd{(datos),}\hlkwc{size} \hlstd{=} \hlkwd{dim}\hlstd{(datos)[}\hlnum{1}\hlstd{]}\hlopt{*}\hlnum{0.75}\hlstd{)}

\hlcom{# CONJUNTO DE ENTRENAMIENTO (selección de columnas)}
\hlstd{train.set}\hlkwb{<-}\hlstd{datos[train.filas,]}
\hlstd{train1}\hlkwb{<-}\hlstd{train.set} \hlopt \hlkwd{mutate_if}\hlstd{(is.numeric,scale)}
\hlkwd{dim}\hlstd{(train.set)}
\end{alltt}
\begin{verbatim}
## [1] 1077    4
\end{verbatim}
\begin{alltt}
\hlcom{# CONJUNTO DE PRUEBA}
\hlstd{test.filas}\hlkwb{<-}\hlkwd{setdiff}\hlstd{(}\hlkwc{x} \hlstd{=} \hlkwd{row.names}\hlstd{(datos),train.filas)}
\hlstd{test.set}\hlkwb{<-}\hlstd{datos[test.filas,]}
\hlstd{test.set1}\hlkwb{<-} \hlstd{test.set} \hlopt \hlkwd{mutate_if}\hlstd{(is.numeric,scale)}
\hlkwd{dim}\hlstd{(test.set)}
\end{alltt}
\begin{verbatim}
## [1] 359   4
\end{verbatim}
\end{kframe}
\end{knitrout}

    \item Entrene los cinco modelos con base en el conjunto de entrenamiento y almacene los correspondientes precios predichos para los automóviles de dicho conjunto.
    \\ \\
    \textbf{Modelo 1}: un modelo de regresión lineal múltiple:
    \[Y= \beta_0 + \beta_1X_1 + \beta_2X_2 + \beta_3X_3 + \epsilon\]
    
\begin{knitrout}
\definecolor{shadecolor}{rgb}{0.969, 0.969, 0.969}\color{fgcolor}\begin{kframe}
\begin{alltt}
\hlstd{modelo1}\hlkwb{<-}\hlkwd{lm}\hlstd{(Price}\hlopt{~}\hlstd{KM}\hlopt{+}\hlstd{Age}\hlopt{+}\hlstd{Weight,}\hlkwc{data}\hlstd{=train.set)}
\end{alltt}
\end{kframe}
\end{knitrout}

    \textbf{Modelo 2}:un modelo de regresión múltiple de grado 3:
    \[Y= \beta_0 + \beta_1X_1 + \beta_2X_2^2 + \beta_3X_3^3 + \epsilon\]
\begin{knitrout}
\definecolor{shadecolor}{rgb}{0.969, 0.969, 0.969}\color{fgcolor}\begin{kframe}
\begin{alltt}
\hlstd{modelo2}\hlkwb{<-}\hlkwd{lm}\hlstd{(Price}\hlopt{~}\hlstd{KM}\hlopt{+}\hlstd{(Age}\hlopt{^}\hlnum{2}\hlstd{)}\hlopt{+}\hlstd{(Weight}\hlopt{^}\hlnum{3}\hlstd{),}\hlkwc{data}\hlstd{=train.set)}
\end{alltt}
\end{kframe}
\end{knitrout}

    \textbf{Modelo 3}:Un modelo ajustado por algoritmo kNN con $k=10$ vecinos más próximos.
\begin{knitrout}
\definecolor{shadecolor}{rgb}{0.969, 0.969, 0.969}\color{fgcolor}\begin{kframe}
\begin{alltt}
\hlstd{modelo3}\hlkwb{<-}\hlstd{FNN}\hlopt{::}\hlkwd{knn.reg}\hlstd{(}\hlkwc{train} \hlstd{= train.set,}\hlkwc{y} \hlstd{=train.set}\hlopt{$}\hlstd{Price,}\hlkwc{k} \hlstd{=} \hlnum{10}\hlstd{)}
\end{alltt}
\end{kframe}
\end{knitrout}
     
     \textbf{Modelo 4}:Un modelo ajustado por algoritmo kNN con $k=10$ vecinos más próximos sobre las variables normalizadas $Z_1$, $Z_2$ y $Z_3$.
\begin{knitrout}
\definecolor{shadecolor}{rgb}{0.969, 0.969, 0.969}\color{fgcolor}\begin{kframe}
\begin{alltt}
\hlstd{modelo4}\hlkwb{<-}\hlstd{FNN}\hlopt{::}\hlkwd{knn.reg}\hlstd{(}\hlkwc{train} \hlstd{=train1 ,}\hlkwc{y} \hlstd{=train1}\hlopt{$}\hlstd{Price,}\hlkwc{k} \hlstd{=} \hlnum{10}\hlstd{)}
\end{alltt}
\end{kframe}
\end{knitrout}

\textbf{Modelo 5}: Un modelo generalizado

\begin{knitrout}
\definecolor{shadecolor}{rgb}{0.969, 0.969, 0.969}\color{fgcolor}\begin{kframe}
\begin{alltt}
\hlcom{# Precios predichos según modelo 1}
\hlstd{yhat_mod1}\hlkwb{<-}\hlkwd{predict}\hlstd{(}\hlkwc{object} \hlstd{= modelo1)}

\hlcom{# Precios predichos según modelo 2}
\hlstd{yhat_mod2}\hlkwb{<-}\hlkwd{predict}\hlstd{(}\hlkwc{object} \hlstd{= modelo2)}

\hlcom{# Precios predichos según modelo 3}
\hlstd{yhat_mod3}\hlkwb{<-}\hlstd{modelo3}\hlopt{$}\hlstd{pred}

\hlcom{# Precios predichos según modelo 4}
\hlstd{yhat_mod4}\hlkwb{<-}\hlstd{modelo4}\hlopt{$}\hlstd{pred}


\hlcom{# Se almacenan junto al conjunto de entrenamiento}
\hlstd{tabla1}\hlkwb{<-}\hlstd{train.set} \hlopt
        \hlkwd{mutate}\hlstd{(}\hlkwc{Price_pred1}\hlstd{=yhat_mod1,}
               \hlkwc{Price_pred2}\hlstd{=yhat_mod2,}
               \hlkwc{Price_pred3}\hlstd{=yhat_mod3,}
               \hlkwc{Price_pred4}\hlstd{=yhat_mod4)}
\hlstd{tabla1}
\end{alltt}
\begin{verbatim}
## # A tibble: 1,077 x 8
##        KM   Age Weight Price Price_pred1 Price_pred2 Price_pred3 Price_pred4
##     <dbl> <dbl>  <dbl> <dbl>       <dbl>       <dbl>       <dbl>       <dbl>
##  1  21684    19   1185 23950      18581.      18581.      20584        3.00 
##  2  62636    22   1255 17950      18680.      18680.      15575        2.28 
##  3  88807    68   1050  8500       8382.       8382.       8505       -0.646
##  4  86714    68   1035  8950       8122.       8122.       8525       -0.582
##  5  81930    76   1070  7750       8021.       8021.       7844.      -0.825
##  6 110287    68   1050  9500       7859.       7859.       9115       -0.330
##  7  69103    68   1035  9750       8551.       8551.       9530       -0.236
##  8 204250    68   1115  7900       6917.       6917.       6305       -1.14 
##  9  29650    55   1025  9950      10835.      10835.      10170       -0.170
## 10  57000    80   1000  7750       6708.       6708.       8255       -0.793
## # ... with 1,067 more rows
\end{verbatim}
\end{kframe}
\end{knitrout}

    \item Estime (y almacene) los correspondientes errores cuadráticos medios de entrenamiento MSEE de los cinco modelos ¿Cuál modelo ajustó mejor al conjunto de entrenamiento?
    
\begin{knitrout}
\definecolor{shadecolor}{rgb}{0.969, 0.969, 0.969}\color{fgcolor}\begin{kframe}
\begin{verbatim}
##    MSE_m3  MSE_m1  MSE_m2    MSE_m4
## 1 1118598 1957205 1957205 129063302
\end{verbatim}
\end{kframe}
\end{knitrout}

    \item Evalúe los modelos entrenados en el paso 4 utilizando el conjunto de prueba y almacene los correspondientes precios predichos para los automóviles de dicho conjunto.

\begin{knitrout}
\definecolor{shadecolor}{rgb}{0.969, 0.969, 0.969}\color{fgcolor}\begin{kframe}
\begin{verbatim}
## # A tibble: 359 x 8
##       KM   Age Weight Price Price_pred1 Price_pred2 Price_pred3 Price_pred4
##    <dbl> <dbl>  <dbl> <dbl>       <dbl>       <dbl>       <dbl>       <dbl>
##  1 72937    23   1165 13750      16448.      16448.      12135        0.825
##  2 41711    24   1165 13950      17091.      17091.      14147        1.08 
##  3 75889    30   1245 18600      17209.      17209.      14225        2.15 
##  4 31461    25   1185 20950      17637.      17637.      19895        2.70 
##  5 18739    28   1185 22000      17593.      17593.      20475        2.82 
##  6 34000    30   1185 22750      16986.      16986.      20025        2.92 
##  7 64359    30   1105 16950      14591.      14591.      15725        1.43 
##  8 43905    29   1170 16950      16552.      16552.      15846.       1.88 
##  9 56349    28   1120 15950      15332.      15332.      14010        1.45 
## 10 41000    22   1100 15500      15998.      15998.      14774.       1.42 
## # ... with 349 more rows
\end{verbatim}
\end{kframe}
\end{knitrout}
\item estime (y almacene) los correspondientes errores cuadráticos medios de prueba MSEP de los cinco modelos.

    
    \item Compare visualmente los MSEE y MSEP de los cinco modelos. A su criterio ¿Cuál modelo escogería para predecir el precio de nuevos autos? Justifique
    \item Con base en el modelo que seleccionó en el punto 7, prediga el precio que tendrán los siguientes tres automóviles con perfiles:
    \begin{table}[H]
        \centering
        \begin{tabular}{|c|c|c|c|}\hline
           \textbf{Automovil}  &  \textbf{KM} & \textbf{Age} & \textbf{Weight} \\ \hline
             1 & 60.000 &30 & 1.300 \\ \hline
            2 & 22.000 & 25  & 1.500 \\ \hline
           3 &  3.000 & 4  & 1.070 \\ \hline
            
        \end{tabular}
        \caption{Caracteristicas de los nuevos autos}
        \label{tab:my_label}
    \end{table}
\end{enumerate}
\begin{thebibliography}{1} % Bibliography - this is intentionally simple in this template
\bibitem{presentation} Ramos David,
\newblock{\em Evaluación de modelos para regresión: Ejemplo},
\newblock  (2020).

\end{thebibliography}
\end{document}
